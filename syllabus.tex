\documentclass[12pt, preprint]{aastex}
% formatting based on 2014 NASA ATP proposal with Jeff Oishi

%%%%%%begin preamble
\usepackage[hmargin=1in, vmargin=1in]{geometry} % Margins
\usepackage{hyperref}
\usepackage{url}
\usepackage{times}
\usepackage{natbib}
\usepackage{graphicx}
\usepackage{amsmath}
\usepackage{amsfonts}
\usepackage{amssymb}
\usepackage{pdfpages}
\usepackage{import}

\hypersetup{
     colorlinks   = true,
     urlcolor    = blue,
     citecolor    = gray
}

%% headers
\usepackage{fancyhdr}
\pagestyle{fancy}
\lhead{ASTR/ATOC 5400}
\chead{}
\rhead{}
\lfoot{Syllabus}
\cfoot{\thepage}
\rfoot{Spring 2021}
% no hline under header
\renewcommand{\headrulewidth}{0pt}

\newcommand{\sol}{\ensuremath{\odot}}
\newcommand{\dedalus}{\href{http://dedalus-project.org}{Dedalus}}

% make lists compact
\usepackage{enumitem}
\setlist{nosep}

%%%%%%end preamble

\begin{document}

{\Huge Introduction to Fluid Dynamics}

Remote MWF: MW asynchronous, F 10:20-11:10am Zoom

\paragraph{Contact info}:\\
Benjamin Brown\\
Laboratory for Atmospheric and Space Physics (LASP)\\
Department of Astrophysical and Planetary Sciences (APS)\\
email: \href{bpbrown@colorado.edu}{bpbrown@colorado.edu}\\
office hours: Slack channel and Zoom \\
~\\
course web presence: Canvas, RC JupyterHub, \\
Zoom (\href{https://cuboulder.zoom.us/j/94833881576
}{https://cuboulder.zoom.us/j/94833881576}) and
Slack (\href{astr5400fluids.slack.com}{astr5400fluids.slack.com}).

\vspace{-0.5cm}

\section*{Why fluids?}

Fluid dynamics underlies many interesting astrophysical problems.  Stars sing in hydrodynamic waves, and by listening to their songs we learn about their interior properties.  Planets form in accretion disks, where swirling flows concentrate or rarify the raw materials of world formation.  The jets of AGN arise from magnetohydrodynamic interactions between plasma flows and magnetic fields, and evolve following basic hydrodynamic processes.  Though fluids are everywhere, our expertise in them is not strong.  When astronomers say ``rotation'' or ``magnetic fields'' do something magical, what they generally mean is ``fluid interactions with rotation and/or magnetic fields'' do something simple.  Here we aim to transform magic into reason.


This course will provide a foundation for your other graduate work, but more importantly, will provide you with a rich toolkit for tackling hard research problems in the future.  We will study fluid dynamics in both nonlinear situations, where we will need to perform numerical simulations, and in linear systems where tools of linear algebra will see practical application.  We will learn how the fluid equations are partial differential equations (PDEs) and review practical solution strategies, computationally and analytically.

I have three major goals in this course.  First, I want you to emerge with a broad set of tools for tackling problems which you will encounter in your future research.  Many research problems require a diversity of approaches, and we'll build fluency with these different approaches here.   Second, I want you to feel prepared for further learning in fluid dynamics.  This course, by nature, must lightly touch many subjects.  Coming out of it however, you will have the fluency to dive deeper into subjects relevant to your research, and familiarity with ways to search for publications in applied math and physics. Third, I want you to emerge from this class with confidence in your computational skills.  We will learn and use python, latex, version control for code, and unix-based systems in the course of our work, as these form the foundations of modern research environments.

\newpage
\section*{Content}
\begin{itemize}
\item INVISCID FLUIDS \emph{derivation and examples of the Euler equations}
\begin{itemize}
\item Continuum hypothesis
\item Eulerian and Lagrangian formulations of fluid flows Inviscid Euler equations
\item Streamlines, streamfunctions
\item Examples of inviscid flows
\item Inviscid flows and numerical solutions
\end{itemize}
\item VISCOUS FLOWS \emph{derivation of the Navier-Stokes equations including the energy equation}
\begin{itemize}
\item Relationships between stress and strain, the stress tensor
\item Navier-Stokes equations: continuity, momentum, energy Reynolds number
\item Transformation to non-inertial frames (e.g. rotating frames)
\item Bernoulli’s equations
\end{itemize}
\item VORTICITY \emph{definition and significance of vorticity in fluid flows}
\begin{itemize}
\item Vortex dynamics
\item Kelvin’s circulation theorem
\item Taylor-Proudman theorem
\item Potential vorticity
\end{itemize}
\item GRAVITY WAVES \emph{concepts required to deal with waves}
\begin{itemize}
\item Linearization
\item Phase and group velocities, concept of the dispersion relation
\item  Gravity waves
\item Deep and shallow water waves
\end{itemize}
\item COMPRESSIBLE FLOWS \emph{compressible fluid dynamics including shocks and sound waves}
\begin{itemize}
\item Thermodynamics of compressible flow 1D flow examples
\item Sound waves
\item Shock waves and jump conditions
\item Weak and strong shocks
\item Sedov solution
\end{itemize}
\item INSTABILITIES \emph{when waves grow}
\begin{itemize}
\item Linearization of the fluid equations and growth rate of perturbations
\item Kelvin-Helmholtz instability
\item Rayleigh-Taylor instability
\item Rayleigh-Benard convection
\end{itemize}
\item BOUNDARY LAYERS \emph{especially important for stars, planets and rockets}
\begin{itemize}
\item Concepts of boundary layers
\item  Self-similar solutions for viscous flows
\item Jets
\item Boundary layer separation
\end{itemize}
\end{itemize}


\section*{Course structure}

\paragraph{Course:} This course will be taught in a fully remote context.  We will use a mixture of asynchronous (two days a week) and synchronous instruction (one day a week, Fridays).    We will also do some hands-on experiments during the class, using common materials from homes and kitchens.  We'll discuss this in advance and come up with individual alternatives if this is a barrier.

\begin{itemize}
  \item Slack workspace: \href{astr5400fluids.slack.com}{astr5400fluids.slack.com}
\end{itemize}

\paragraph{Asynchronous Class:} The asynchronous material is a crucial part of your learning in the course.  Asynchronous instruction will likely include a mixture of recorded short lectures and guided interactive exercises using  Jupyter notebooks.  We'll adapt this as we proceed through the semester and see what is working better or worse.  You will be expected to (asynchronously) interact with me on each of our asynchronous days (MW) via our shared Slack workspace and our Canvas page.

\paragraph{Synchronous Class:} Our synchronous classes will be conducted via Zoom on Fridays.  These days will likely be less-lecture, more hands-on work on projects and experiments and guided discussion.  We will have readings from the textbook, and possibly online material, that I expect you to have studied before class.  These will be communicated to you, and I will expect you to please come to class with questions.  We will all learn more through interaction and discussion.  I won't know all of the answers.  When I don't, I'll find out and bring them to you.  If you need to miss class, please discuss this with me in advance so we can make accommodations.

\paragraph{Homework:} Early in the semester, there will be homework almost every week, handed out and collected on Wednesdays.  Most homeworks will involve some computational aspect, using resources at CU's Research Computing.  Please collaborate and work together, but please write up your work (and write your code) individually.  Homework will be written up in Jupyter notebooks and will be handed in via our shared RC Jupyterhub workspace.


\paragraph{Final Exam:} there will be a final exam.  It will be a timed, asynchronous, and will be in April.

\paragraph{Final project:} each of you will do a final project studying an interesting fluids problem. These can be computational, experimental, or observational (of fluids in our everyday world). You will give a short, AAS-style talk (5~min presentation, 3~min questions) about your approach and your findings. These will be done in the format of current online meetings to give you practice in these formats in a safe space.  This will occur during Finals week. This is in addition to the final exam (above).

\paragraph{Note taking:}
This class does not have a single topic, and our textbook does not capture everything we will cover.  Instead, we are covering a variety of subjects, and this can make it difficult to figure out what exactly we covered when it’s time to study for your comprehensive exam.

To help remove this concern, we as a class are going to generate a set of notes from the lectures. Each student will take notes on one or more classes and type those notes up a presentable fashion in whatever composer you prefer (Latex, word, Jupyter, etc.).  When they are in sharable form, you will e-mail them to me as both a PDF and also your original document and then I will upload them on our shared Canvas site.  Notes are due within two days of class (e.g., if you take notes Monday, please get them to me Weds evening), with an additional day allowed if you take notes Friday (due Tues evening), and notes will be taken for our asynchronous days as well as our synchronous meetings.

Your notes do not need to be works of literature nor art.  I will however return your notes to you for revision if they are not clear enough for your fellow students to follow.  If there is a minor clarity or typo error, I will revise the original document myself directly, but if the confusion is more substantial, they will come back to you for revision.

The exact number of classes you need to cover will depend on how many of us there are, but I’m estimating it at 1-2 classes each, and we will proceed through the roster alphabetically.  Overall, note-taking is worth about 1 homework of credit and I expect about that level of effort ($\sim 6$ hours total).

\paragraph{Grading:} about 50\% of your grade will comes from
assignments and note-taking, 20\%~from the final exam, and 20\% from the final project.  The last 10\% is from participation, especially on asynchronous days.



\begin{table}[h]
  \begin{center}
\begin{tabular}{lc}
  week 1    & Static solutions to fluid equations \\
  weeks 2--3 & Thermodynamics and energy equations \\
  week 4 & Vorticity \& Kelvin circulation theorm \\
  week 5 & Bernoulli \& non-viscous flows \\
  weeks 6-7 & Viscous flows \\
  week 8 & Non-dimensionalization \& instability \\
  weeks 9-10 & Linear waves \\
  week 11 & Non-constant coefficient atmospheres \\
  week 12 & Rotating fluid dynamics \\
  week 13 & Geophysical and Astrophysical Fluid Dynamics
\end{tabular}
\caption{Approximate flow of the course; subject to change as we progress depending on where we have understanding or need more instruction.}
\end{center}
\end{table}

\section*{Technical Computing}

You will need access to a computer (preferably a unix/linux or Mac), and during some of our synchronous days we will be doing hands-on exercises using your machines.  In this course, we will do all of our work in Python.  Why  Python? Traditionally, many astronomers, especially here in Colorado, have used IDL.  IDL  itself is a closed source platform, and it is difficult to execute parallel processing in IDL.  There is  a growing trend away from IDL and towards Python.  Python is a free, open source, high-level  interactive interpreted computing and scripting environment. Python is used for much more than  scientific computing. One can easily wrap existing C, C++, or Fortran codes. Within the  astronomical community, it is used for everything from telescope observing scripts to quick  interactive data visualization, to sophisticated and complex analysis pipelines with hundreds of  thousands of lines of code.  In this course you will use and learn \verb+python3+  (version 3); there are subtle  array promotion differences between \verb+python2+ and \verb+python3+, and portions of our numerical work will leverage the \dedalus{} pseudospectral framework, which itself uses python3.  To reduce the difficulty of deploying python on different laptops, we will use shared resources on Research Computing's JupyterHub.



\section*{Books to learn from}

\subsection*{Fluids}
\begin{itemize}
\item \emph{The Physics of Fluids and Plasmas}, Choudhuri, 1998 \\
  \textbf{Required course textbook;} a good, broad introduction to fluid dynamics with an astrophysical slant to the material.  Think more helioseismology, supernova shock waves, and convection, less pipe flow turbulence and wings (though some examples of those too).  Covers basic plasma physics and magnetohydrodynamics.

\item \emph{Hydrodynamic and Hydromagnetic Stability}, Chandrasekhar, 1961
\textbf{Required course textbook;} covers instabilities (like convection) in astrophysically important systems.  Critical text to have in your library.
\\[0.5cm]

\item \emph{Fluid Mechanics} Landau \& Lifshitz 1959 (1966 for $3^{rd}$ ed)\\
  Volume 6 (of 9) in Landau \& Lifshitz's sweeping ``Course of Theoretical Physics''.
  Everything is here, if you can understand it.

\item \emph{Physical Fluid Dynamics} Tritton 1988 \\
Beautiful book on incompressible flow with a focus on geophysical
fluid dynamics.  Highly recommended by one of the best fluid
dynamicists I know.

\item \emph{An Introduction to Astrophysical Fluid Dynamics} Thompson 2006 \\
Compact book with strong focus on astrophysical fluid dynamics; close
second runner for main course textbook.  Great material on stellar
oscillations (waves).

\end{itemize}

\subsection*{Math}
\begin{itemize}
\item \emph{A First Course in Numerical Methods}, Ascher \& Greif, 2011 \\
  A broad introduction to linear algebra, numerical techniques,
  differential equations, etc.  Good reference for understanding
  packaged library routines (e.g., QR factorization, Krylov methods, etc.)
  and the code examples, though in Matlab, are helpful.  Freely
  available as a PDF from a campus IP address
  (\url{http://epubs.siam.org/doi/book/10.1137/9780898719987}).
\end{itemize}

\subsection*{Numerical techniques and coding}
\begin{itemize}
\item O’reilly, “Linux in a Nutshell,” or “Unix in a Nutshell,” is a comprehensive linux/unix command,
shell, and text editor reference.
\item  Press et al, “Numerical Recipes 3rd Edition.” In addition to offering specific numerical
algorithms, this book contains excellent introductory text on statistics and data analysis
techniques.  Beware the restrictive licensing clauses in this book.
\end{itemize}

\subsection*{Python}
\begin{itemize}
\item Basic Python tutorial: \url{http://docs.python.org/tutorial/}
\item Free (and good) online text: \url{https://openbookproject.net/thinkcs/python/english3e/}
\item Scientific computing using Python: Scipy and Numpy: \url{http://scipy.org/}
\item Using Python for Interactive Data Analysis in Astronomy: \url{http://stsdas.stsci.edu/perry/pydatatut.pdf}
\item Interactive computing with Python using Jupyter: https://jupyter.org
\item  Matlab-like interactive plotting with Python: http://matplotlib.sourceforge.net/
\item UC Berkeley Astronomy Department’s Python Boot Camp:  \url{https://sites.google.com/site/pythonbootcamp/}
\item O’Reilly’s “Learning Python,” 3rd edition: http://oreilly.com/python/
\item Resources for installing a base python system (with python, numpy, scipy, etc.):
\begin{itemize}
\item Dedalus stack: \url{http://dedalus-project.readthedocs.io/en/latest/installation.html}
\item Anaconda: \url{https://www.continuum.io/downloads}
\end{itemize}
\end{itemize}

\subsection*{Latex}

\begin{itemize}
\item \emph{A (Not So) Short Introduction to \LaTeXe},
  Oetiker\\
  \url{https://www.ctan.org/tex-archive/info/lshort/english/} or \\
  \url{https://tobi.oetiker.ch/lshort/}\\
   Good go-to reference for using Latex.  Freely available as PDF.

\end{itemize}

\end{document}
